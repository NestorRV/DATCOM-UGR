\documentclass[11pt]{article}
\usepackage{amsmath}
\usepackage{graphicx}
\usepackage{hyperref}
\usepackage[utf8]{inputenc}
\usepackage[spanish]{babel}
\usepackage[margin=2.5cm]{geometry}
\usepackage{amsfonts}
\usepackage{listings}
\usepackage[T1]{fontenc}
\usepackage{float}
\usepackage{subfig}
\usepackage{textcomp}

\title{Trabajo 5 - Detección de peatones.}
\author{Néstor Rodríguez Vico. DNI: 75573052C - \href{mailto:nrv23@correo.ugr.es}{nrv23@correo.ugr.es}}
\date{\today}


\lstdefinestyle{bash_style}{
	language=MatLab,
	frame=single,
	xleftmargin=.25in,
	upquote = true,
	basicstyle=\scriptsize,
	breakatwhitespace=false,         
	breaklines=true,                 
	captionpos=b,                    
	keepspaces=true,                 
	numbers=left,                    
	numbersep=5pt,                  
	showspaces=false,                
	showstringspaces=false,
	showtabs=false,                  
	tabsize=2,
	upquote=true
}

\lstset{style=bash_style}

\begin{document}

\maketitle

\setlength{\belowdisplayskip}{5pt} 
\setlength{\belowdisplayshortskip}{5pt}
\setlength{\abovedisplayskip}{5pt} 
\setlength{\abovedisplayshortskip}{5pt}

\section{Solución.}

Junto a este documento se entrega el fichero \textit{hog.m} donde se ha implementado el descriptor y se ha hecho uso de una \textit{SVM} para el proceso de clasificación. Los pasos usados para implementar el descriptor han sido:

\begin{enumerate}
	\item Calculamos los gradientes en X y en Y.
	\item Calculamos la orientación y la magnitud del gradiente en cada píxel.
	\item Dividimos la imagen en celdas. Para cada celda calculamos el histograma de gradientes ponderado, tal y como viene explicado en el PDF.
	\item A continuación, agrupamos las celdas en bloques y creamos un ``histograma'' a nivel de bloque como la concatenación de los histogramas de las celdas que conforman el bloque. Finalmente, normalizamos el histograma del bloque saturando los valores que superan el \textit{0.2} a \textit{0.2}.
	\item Una vez tenemos los histogramas normalizados, los concatenamos para tener un único vector que describa la imagen.
\end{enumerate}

\section{Resultados.}

Para el conjunto de \textit{trian}, la matriz de confusión es la siguiente:

\begin{table}[H]
	\centering
	\begin{tabular}{c|c}
		41 & 1 \\ \hline
		0 & 318
	\end{tabular}
\end{table}

Obtenemos un \textit{accuracy} de \textit{0.9972}. Para el conjunto de \textit{test}, la matriz de confusión es la siguiente:

\begin{table}[H]
	\centering
	\begin{tabular}{c|c}
		8 & 0 \\ \hline
		1 & 81
	\end{tabular}
\end{table}

Obtenemos un \textit{accuracy} de \textit{0.9889}.

\end{document}